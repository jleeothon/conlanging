\section{Adjectives}

Adjectives inflect according to their grammatical function, number, and case.

In the \textbf{predicative} form, the adjective does not inflect.

In the \textbf{attributive} form:

\begin{itemize}
\item the suffix \textit{-a} is always added,
\item the prefix \textit{ga-} is added if the noun precedes the adjective.
\end{itemize}

Examples:

\begin{itemize}
\item Þe kat is hongrig.
\item Þe hongrigę kat.
\item Þe kat gąhongrigę etęt.
\item Ek sehęw þen hongrigę kat.
\item Ek sehęw þen kat gąhongrigę.
\end{itemize}

\subsection{Comparative and superlative}

The comparative degree is derived by umlauting the last non-weak vowel and adding the suffix \textit{-ir}. Comparative adjectives are declined as regular adjectives. Compared objects are in nominative case, and reference objects in the genitive case.

The superlative degree is derived by umlauting the last non-weak vowel and adding the suffix \textit{-ist}. Superlative adjectives are declined as regular adjectives.

Examples:

\begin{itemize}
\item Þe mús is klaïnir þes kuwes, \textit{or} Þe mús þes kuwes klaïnir is (\textit{The mouse is smaller than the cow}).
\item Þe bláw hwal is þe graütiste fisk in þen mór (\textit{The blue whale is the biggest fish in the ocean}).
\item Þe bl\H{a}wiste hwal þes bl\H{a}wistes hájs bl\H{a}wir is. (\textit{The bluest whale is bluer than then bluest shark}).
\item Þej bl\H{a}wistej hwäl þer bl\H{a}wister h\H{a}js bl\H{a}wir sűn. (\textit{The bluest whale is bluer than then bluest shark}).
\end{itemize}

Irregular comparatives and superlatives include:

\begin{itemize}
\item gód > bätir > bäst
\item feul > maïr > maïst
\end{itemize}
