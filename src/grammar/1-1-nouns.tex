\section{Nouns}

\subsection{Noun inflection paradigm}

Nouns declinate for nominative-accusative and genitive, and singular and plural.

r-stems:

\begin{center}
\begin{tabular}{ll}
Singular & Plural \\
\hline
bróþär  & brőþäriiu \\
fádär   & fa̋där \\
dohtär  & döhtär \\
módär   & mődär \\
swéstär & swéstärĕn \\
\end{tabular}
\end{center}

The genitive is formed by attaching -s at the end of a noun after declining it for its number (or -es after ⟨z⟩ or ⟨s⟩). For example:

\begin{center}
\begin{tabular}{l|l}
þe kat > þes kats     & þej käten > þer kätens  \\
þe hund > þes hunds   & þej hünden > þer hündens \\
þe wáter > þes wáters & þej w\H{a}t'ren > þer w\H{a}t'rens \\
þe leuht > þes leuhts & þej leühten > þer leühtens \\
þe aug > þes augs     & þej aügen > þer aügens \\
\end{tabular}
\end{center}
