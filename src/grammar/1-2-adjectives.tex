\section{Adjectives}

Adjectives inflect according to their grammatical function, number, and case.

In the \textbf{predicative} form, the adjective does not inflect.

In the \textbf{attributive} form, a suffix is attached to the adjective
according to case and number. An attributive adjective usually \textit{precedes} the
noun it modifies.

When the adjective \textit{follows} its noun, it is prefixed with ``ge-''. This belongs to a poetic register.

Weak inflection paradigm:

\begin{center}
\begin{tabular}{|c|cc|}
\hline
& Singular & Plural \\
\hline
Nominative & -e (see note)  & -en \\
Accusative & -en & -en \\
Genitive   & -es & -en \\
\hline
\end{tabular}
\end{center}

Strong inflection paradigm:

\begin{center}
\begin{tabular}{|c|cc|}
\hline
& Singular & Plural \\
\hline
Nominative & -e (see note)  & -ej \\
Accusative & -en & -em \\
Genitive   & -es & -er \\
\hline
\end{tabular}
\end{center}

Note: in the weak and strong inflection paradigms, the nominative final -ĕ is dropped when the adjective succeeds the noun i.e. \textit{þe hongrigĕ hond} but \textit{þe hond g'hongrig} (``the hungry dog'' and ``the dog a-hungry'').

Examples:

\begin{itemize}
\item Þe kat is hongrig (predicative).
\item Þe hongrige kat (before, sg., nom.).
\item Þe kat g'hongrig ëtet (after, sg., nom.).
\item Þe hongrigen wolf ëtet þen kat (before, sg., acc.).
\item Þe wolf gehongrig þen kat ëtet (after, sg., acc.).
\item Þej hongrigej hönd
\end{itemize}

\subsection{Comparative and superlative}

The comparative degree is derived by umlauting the last non-weak vowel and adding the suffix \textit{-ir}. Comparative adjectives are declined as regular adjectives. Compared objects are in nominative case, and reference objects in the genitive case.

The superlative degree is derived by umlauting the last non-weak vowel and adding the suffix \textit{-ist}. Superlative adjectives are declined as regular adjectives.

Examples:

\begin{itemize}
\item Þe mús is klaïnir þes kuwes, \textit{or} Þe mús þes kuwes klaïnir is (\textit{The mouse is smaller than the cow}).
\item Þe bláw hwal is þe graütiste fisk in þen mór (\textit{The blue whale is the biggest fish in the ocean}).
\item Þe bl\H{a}wiste hwal þes bl\H{a}wistes hájs bl\H{a}wir is. (\textit{The bluest whale is bluer than then bluest shark}).
\item Þej bl\H{a}wistej hwäl þer bl\H{a}wister h\H{a}js bl\H{a}wir sűn. (\textit{The bluest whale is bluer than then bluest shark}).
\end{itemize}

Irregular comparatives and superlatives include:

\begin{itemize}
\item gód > bätir > bäst
\item feul > maïr > maïst
\end{itemize}
